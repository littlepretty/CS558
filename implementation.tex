\section{Implementation}

\subsection{TensorFlow 101}
TensorFlow~\cite{TensorFlow} is an open-source software library for machine learning developed by the Google Brain Team.
Multidimensional data arrays are presented as tensor data structure in TensorFlow.
The user of the library can constructs a graph to implement any general machine learning algorithm.
Nodes in the graph represent mathematical operations between tensors,
such as add, multiply, softmax and dropout.
Graph edges represent the flow of tensors between nodes.
The computation graph based architecture allow researchers to run or train neural networks
on one or more CPUs or GPUs with unified API.

For general classification task, input $X$ and label $Y$ are defined as \textbf{placeholder}s
and feed into the computation graph at running time using a dictionary.
The graph of a typical deep learning model have three parts.
The \textbf{inference} graph should be built so that output predictions could be returned as tensor.
For example, in the multilayer perceptron case, inference graph contains all iterative
computation~\ref{Equ:MLPFeedForward1}-\ref{Equ:MLPFeedForward2} in the feed-forwarding steps.
The \textbf{loss} graph should compute the loss function defined by particular models.
Usually it is either cross-entropy or mean-squared error averaged across the batch data.
The loss graph will be optimized, usually minimized, by the \textbf{train} part.
This optimization can be conducted by various optimizing algorithms, such as gradient descent,
Momentum, RMSProp.
After sufficient steps of batch training, we can evaluate the trained model with inference
graph and compare the predictions with the test dataset labels.
TensorFlow also provides various useful utilities for training models and running experiments.
Using a Saver, we are able to checkpoint the training process used to later restore a model for
further training or evaluation.
User can also use Summary nodes to log the snapshot of interest variables, which can be
automatically visualized by TensorBoard.

\subsection{NetLearner}
We provide a Python library NetLearner that wraps up several deep learning models on the basis of TensorFlow.
NetLearner modularizes multilayer perceptron, restricted Boltzmann machine, sparse autoencoder
and masking-noise autoencoder, all of which are used to perform the 5-class classification
on the NSL-KDD dataset.

For the multiple layer perceptron, we tried a 4-hidden-layer network with
very wide size in each layer, several hundreds for each layer.
The accuracy on training set is very exiting, usually more than 96\%.
However, its performance on test dataset is not satisfactory at all.
Instead we found out that a single hidden layer with only forty neurons has terrific accuracy.
It is trained with stochastic gradient descent (SGD) for 20 epochs and batch size 100.
During the training, learning rate decays from 0.1 exponentially with the base of 0.32.
We did not include regularization in the model, but did apply dropout of keep probability 0.8.
We denote this approach as MLP and show its detailed results in the later section.

We build a RBM with 200-hidden units to perform unsupervised feature learning first on the dataset.
The learned features are then fed into a simple softmax regression classifier.
We trained the RBM using CD1 (contrastive divergence using one full step to get the negative data),
with batch size 10 for 40 epochs.
The learning rate is initialized at 0.01 and decay exponentially with the base of 0.64.
We denote this combination of RBM and softmax regression as RBM in the later section.

We also implemented the self-taught learning architecture proposed in~\cite{STL-NIDS, SparseAE},
adopting sparse autoencoder as the unsuperviesed feature learner.
The learnt features will then be used for classification by a Softmax regression classifier.
We contact the author of~\cite{STL-NIDS} so that we can reproduce their implementation
with the same hyper-parameters.
For example, the hidden layer size of the sparse autoencoder is 64;
the sparsity value $\rho$ is 0.25.
Different from~\cite{STL-NIDS}, we found that using regularization in
neither autoencoder nor softmax regression is helpful.
So we didn't include regularization term in the both autoencoder and softmax cost function.
The autoencoder is trained with SGD for 1000 epochs and batch size 5000.
Different from MLP, we used Adam optimizer during the training.
The learning rate starts at 0.01 and decay exponentially with base of 0.6.
We denote this approach as SAE and report its performance in the later section.

As a variation to the sparse autoencoder based self-taught learning arhictecture,
we explore what will the performance be if we replace sparse autoencoder with denosing autoencoder.
We simple use dropout to emulate the masking noise to build masking noise autoencoder,
in which input is randomly masked out with keep probability of 0.4.
The size of the denoising autoencoder is 100.
The autoencoder is trained with SGD for 1000 epochs and batch size 5000.
We trained denoising autoencoder in the same way as we trained sparse autoencoder.
The result of this approach is labeled as DAE in the later section.

One thing to notice is that we use the same seed to randomly initialized the weights and biases
of the softmax regression classifier such that the learnt features from RBM, sparse autoencoder and
denoising autoencoder are comparable.
For the same reason, all the softmax regression classifiers used by RBM, sparse autoencoder and
denoising autoencoder are trained with Adam optimizer of batch size 100 for 100 epochs,
with exponentially decay learning rate starting at 0.01, with dropout of keep probability 0.8.
