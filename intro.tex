\section{Introduction}

Network intrusion detection system is the essential security technology that
aims to protect a computer network intellengently and automatically.
As either a hardware device or software application,
it monitors a network for malicious activities or policy violations.
By intercepting and analysing the bidirection traffics through the network,
it raises alarm if intrusion, attack or violation are observed.
There are two general approaches to detect intrusions.
In signature based intrusion detection, e.g. SNORT~\cite{Snort},
rules for specific attacks are pre-installed in the system.
It report suspecious traffic when the traffic matches any signature of known attacks.
The major drawback of signature matching approach is that
it is only effective for previously detected attacks that have an identifiable signature.
As a result, signature database needs to be manually updated whenever a new type of attack
is discovered, with significant effort, by the network administrator.
Anomaly detection based approach overcomes the these limitations by adopting a certain
type of machine learning technique to model the trustworthy network activities.
Traffics that significantly deviates from the built model are treated as malicious.
This idea have been shown to be able to detect unknown or novel attacks~\cite{NSL-KDD, STL-NIDS}.
However, if the built model for normal traffics are not generalized enough,
anomly based approach will treat unforeseen normal traffic as malicious,
suffering from high false positive.

In this project, we follow the anomly based idea, and tries to enhance it with the
state-of-art machine learning technology, e.g. several deep learning architectures.
Specifically, we take the task of offline network intrusion detection for the well-known
NSL-KDD dataset~\cite{NSL-KDD}.
Several deep learning approaches are first invetigated and discussed.
We then present NetLearner, an implementation of all of the investigated approaches,
on the basis of TensorFlow.
The detection performance are meatured in accuarcy, precision, recall and F-Score.
